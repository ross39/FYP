\chapter{Introduction}

Fake news (also known as junk news, pseudo-news, or hoax news) is a form of news consisting of deliberate disinformation or hoaxes spread via traditional news media
(print and broadcast) or online social media.Digital news has brought back and increased the usage of fake news, or yellow journalism.The news is then often reverberated
as misinformation in social media but occasionally finds its way to the mainstream media as well.

The main aim behind fake news is to mislead and sway public opinion surrounding an agency, entity, or person, and/or gain financially or politically, often using sensationalist,
dishonest, or outright fabricated headlines to increase readership. Similarly, clickbait stories and headlines earn advertising revenue from this activity Fake news undermines proper
news coverage and makes it more difficult for journalists to cover signifigant news stories. The term `fake news' became very popular in 2016 when during and after his presidential
campaign and election, Donald Trump popularized the term `fake news' in this sense, regardless of the truthfulness of the news, when he used it to describe the negative press coverage of himself.

“Fake news” has acquired a certain legitimacy after being named word of the year by Collins, following what the dictionary called its “ubiquitous presence” over the last 12 months. At present technology companies are fighting an epedemic of fake news. Facebook has deleted 3.39 billion fake accounts from October 2018 to March 2019. Whatsapp is deleting 2m accounts per month. Fake news spread through Whatsapp in India is responsible for 30 mob lynchings that were said to have been triggered by incendiary rumours spread using the app. 

The dynamics and influence of fake news on Twitter during the 2016 US presidential election remains to be clarified, however, a study found that in the five months preceding the election, a dataset of 171 million tweets were gathered and a subset of 30 million tweets, from 2.2 million users, which contain a link to news outlets was gathered. The study found that of those 30 million tweets, approximately 25\% of the news outlets linked containted outright false or extremely biased information\cite{bovet2019influence}. It's important to remember that this is just in relation to American politics.



